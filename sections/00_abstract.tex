\section*{TÓM TẮT (ABSTRACT)}
\addcontentsline{toc}{section}{TÓM TẮT}

Báo cáo này trình bày kết quả nghiên cứu và thực nghiệm về việc nhận diện trạng thái cảm xúc của người học (Boredom, Confusion, Engagement, Frustration) trong môi trường học tập trực tuyến. Sử dụng tập dữ liệu chuẩn DAiSEE, nhóm nghiên cứu đề xuất phương pháp tiếp cận mới dựa trên Mô hình Ngôn ngữ-Thị giác Lớn (Vision-Language Model - VLM), cụ thể là \texttt{Qwen2.5-VL-7B-Instruct}, để trích xuất đặc trưng ngữ nghĩa mức cao từ video người học. Các đặc trưng này sau đó được sử dụng để huấn luyện một bộ phân lớp MLP (Multi-Layer Perceptron). Bên cạnh đó, nhóm cũng áp dụng kỹ thuật tăng cường dữ liệu sử dụng các bộ dữ liệu khuôn mặt tĩnh (Facial Expression Data) để khắc phục vấn đề mất cân bằng dữ liệu nghiêm trọng của DAiSEE. Kết quả thực nghiệm cho thấy phương pháp đề xuất đạt được độ chính xác khả quan trên một số lớp, tuy nhiên vẫn còn thách thức lớn đối với các lớp thiểu số do bản chất mất cân bằng của dữ liệu. Báo cáo phân tích chi tiết hiệu năng của các cấu hình mô hình khác nhau và đề xuất các hướng cải tiến trong tương lai.

\noindent\textbf{Từ khóa:} \textit{Engagement Recognition, Vision-Language Models, Qwen2.5-VL, DAiSEE, Imbalanced Learning.}
